\documentclass[12pt, a4paper]{article}
\usepackage[top=3.5cm, bottom=2.5cm, left=2.5cm, right=2.5cm]{geometry}

\usepackage{array}
\usepackage{longtable}
\usepackage{graphicx}
\usepackage[square, numbers]{natbib}


\begin{document}
	\title{An Introduction to \LaTeX and Git}
	\author{Benjamin Yee 45425108}
	\maketitle
	
	\tableofcontents
	
	\section{\LaTeX Introduction}
	
	Latex [1] is a document preparation system for high-quality typesetting by which
	a writer uses plain text instead of formatted text you have seen in Microsoft
	Word. Scientific and technical documents can benefit from the \LaTeX features as
	it easily facilitates producing very high quality and complex documents through,
	including but not limited to, exceptional referencing capabilities and management of tables, figures, or bibliographic. In addition, you can find freely available
	templates for any document.
	
	\section*{Exercises}
	There is a 30-minute introduction to learn \LaTeX . Follow Learn \LaTeX in 30
	Minutes tutorial, and you will learn the basics helping you with the following
	exercises for the first part of the workshop.
	To practise \LaTeX , you can use different tools either offline (e.g., TexWorks)
	or online. In this workshop, we are after the online one called Overleaf as the
	well-known platform for working with \LaTeX . Before following the tutorial, you
	need to create an account on Overleaf. You may use your personal email or
	your university email address (ending @students.mq.edu.au). Since Macquarie
	University has an enterprise subscription with Overleaf, you will have full access
	to the Overleaf by using your university email address.
	
	\subsection{Task 1.1: Create your first \LaTeX document}
	You are required to create a (\LaTeX) document for the first part of the workshop
	that contains the following items:
	
	\begin{figure}[h]
		\centering
		\includegraphics[width=0.5\textwidth]{"DALL·E 2023-01-22 20.33.05 - a photo of a yellow labrador plush toy on a bookshelf.png"}
	\end{figure}


	\begin{longtable}[h!]{| m{4cm} | m{11cm} |}
		\hline \multicolumn{2}{|r|}{\textit{Continued on next page}} \\ \hline \endfoot
		\hline \multicolumn{2}{|r|}{\textit{Continued from previous page}} \\ \hline \endhead
		\endfirsthead
		\endlastfoot
		\hline
	
		\textbf{Title}		&		\textbf{Description}		\\ \hline
		Yes			&		Means not no	\\	\hline
		git command	&	git description	\\ \hline
		
	\end{longtable}
	
	
	\section{ds-sim}
	ds-sim [4] is an open-source, language-independent and configurable distributed
	systems simulator. It is designed to perform a quick and realistic simulation of
	job scheduling and execution in distributed systems, such as computer clusters
	and (cloud) data centres.
	
\end{document}